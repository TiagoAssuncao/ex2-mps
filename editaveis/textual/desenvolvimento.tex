\chapter[Desenvolvimento ]{Desenvolvimento}

Nesta sessão iremos responder todas as questões propostas anteriormente, sendo
que cada subsessão será responsável pela sua respectiva questão.

\section{Primeira Questão}
\label{sec:Primeira Questão}

TO DO

\section{Segunda Questão}
\label{sec:Segunda Questão}

Um exemplo de processo muito comum para o ambiente de desenvolvimento de software
é a implementação de uma história de usuário. Esta, como entrada, contém o
requisito solicitado pelo usuário e como saída, o incremento de software produzido.
A atividade é desempenhada por aquele que tem o papel de desenvolvedor, executando
assim, a implementação do requisito.

A política utilizada é que o desenvolvedor só poderá declarar a história como
concluída quando todos os critérios de aceitação estiverem cumpridos, como:
cobertura de testes, enquadramento em folha de estilo, entre outros pontos. Como
método, o desenvolvedor utilizou metodologia ágil.

\section{Terceira Questão}
\label{sec:Terceira Questão}

Um exemplo de processo imaturo é o cenário onde um desenvolvedor recebe o requisito
do cliente e o implementa, fazendo-o de maneira cíclica e sem priorização.
Neste caso, não existe nenhum controle sobre as saídas ou entradas do processo,
apenas é feito.

Um processo maduro, como exemplo, é o produto que as disciplinas de MDS/GPP
desenvolvem durante o semestre, onde é feito o planejamento das entregas, medições
dos resultados, dos incrementos de software.
